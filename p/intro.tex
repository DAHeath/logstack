\section{Introduction}\label{sec:intro}

Secure two party computation (2PC) of programs representable as Boolean circuits can be efficiently achieved using garbled circuits (GC).
%
However,  circuit-based MPC in general is problematic because conditional
control flow does not have an efficient circuit representation:
in the cleartext program, only the taken execution is computed whereas in
the circuit \emph{all} branches must be computed.
%While this creates obvious relative overheads in conditional statements, the most significant impact is on general control flow, such as a sequence of loops.

%
Until recently, it was assumed that the players must not only compute
all branches, but also transmit a string of \emph{material} (i.e., the garbled circuit itself) 
proportional to the entire circuit.  
Since communication is the GC bottleneck, transmitting this large string was
problematic for programs with conditionals.

Stacked Garbling~\HK, which we %(and \HK) 
interchangeably call Stacked Garbled Circuit (SGC), shows that
expensive branching-based communication is unnecessary: the players need only
send enough material for the single longest branch. This single
piece of \emph{stacked} material can be re-used across all conditional branches,
substantially reducing communication cost.
%
Unfortunately, this improvement comes with one important downside:
SGC requires the players to compute more than they would have without stacking.
In particular, for a conditional with $b$ branches, the \HK GC
generator must evaluate each branch under encryption $b-1$ times, and
hence pays $O(b^2)$ total computation.
In contrast, standard garbling uses computation linear in the number
of branches.

In this work, we present a new construction for SGC that incurs
only $O(b \log b)$ computation for both players while
retaining the important communication improvement of \HK.
%
The construction also features improved space complexity: While \HK
requires the generator to store $O(b)$ intermediate garblings, our
constructions requires only $O(\log b)$ space.
%
Finally, the construction features low constants, and hence opens the
door to using SGC even in the presence of very high branching factors
without computation becoming prohibitive.
% Next, we argue that
% efficient  support for high branching factor has wide applications in
% MPC.


\subsection{A Case for High Branching Factor}
\label{sec:motivationHighB}

Branching is ubiquitous in programming, and our work significantly
improves the secure evaluation of programs with branching.
Moreover, the efficient support of \emph{high branching factor}
is more important than it may first appear.

Efficient branching, such as what is
achieved by our work, enables optimized handling of \emph{arbitrary
control flow}, including arbitrary repeated and/or nested loops.
%
Specifically, we can repeatedly \emph{refactor} the source program
until the program is a single loop whose body conditionally dispatches
over straightline fragments of the original program\footnote{%
  As a brief argument that this is clearly possible, consider that a
  CPU has this exact structure: in this case the `straightline
  fragments' are the individual instruction types handled by the CPU.
}.
However, these types of refactorings often lead to conditionals with
high branching factor.

As a simple example,
consider a program $P$ consisting of a loop $L_1$ followed by a loop
$L_2$.  Assume the upper bound on runtime (total number of loop
iterations $T$) of $P$ is known, as is usual in MPC.
For security, we must protect the number of iterations $T_1$ of $L_1$
and $T_2$ of $L_2$.
Implementing such a program with standard Yao GC requires us to
pay double of the cost achievable with stacked garbling: we must
execute loop $L_1$ $T$ times followed by executing $L_2$ $T$ times.
At the same time, SGC can simply execute  $\stack(L_1, L_2)$ $T$
times, a circuit with a significantly smaller garbling. This observation corresponds to the
following refactoring:
\[{\tt while (e_0) \{ s_0 \} ; \ while (e_1) \{ s_1 \}}
\longrightarrow {\tt while (e_0 \lor e_1) \{\ if  (e_0) \{s_0\}\  else\  \{s_1\}\ \}} \]
where ${\tt s_i}$ are nested programs and $e_i$ are predicates on program
variables\footnote{%
  To be pedantic, this specific refactoring is not always
  valid: $\mathtt{s_1}$ might mutate variables used in
  $\mathtt{e_0}$. Still, similar, yet more notationally complex,
  refactorings are always legal.
}.
The right hand side is friendlier to SGC, since it
substitutes a loop by a conditional.
Now, consider that $s_0$ and $s_1$ might themselves have conditionals
that can be flattened into a single conditional with all branches.
By repeatedly applying such refactorings, even modest
programs can have conditionals with high branching factors.
High-performance branching, enabled by our approach, will allow the
efficient and secure evaluation of such programs.

% In this work, we do not attempt to systematize the possible SGC-based
% optimizations of generic flow control.
We do not further explore the space of program refactorings while
considering SGC.
% consideration 
% , but instead provide a
% powerful approach for handling conditionals, even if they have high
% branching factor.
However, we firmly believe that SGC is an essential tool that will
enable research into this direction, and further believe, as argued
above, that performance in the presence of high branching factor is
essential.




\subsection{\HK and its $O(b^2)$ computation}
\label{sec:bsquaredcost}

Our approach is similar to \HK: we also stack material in
order to decrease communication.
The key difference is our reduced computation.
It is thus instructive to review \HK,
focusing on the source of its quadratic computation.

The key idea of SGC is that the circuit generator \G\ garbles,
starting from seeds, each branch $\cir_i$.
He then \emph{stacks} these $b$ garbled circuits, yielding only a
single piece of material proportional to the longest branch: $M =
\bigoplus_i \gcir_i$. (Note,~\HK, as do we in this work,  pad each GC material $\gcir_i$ with randomness before stacking.  This ensures all $\gcir_i$ are of the same length.)
Because the garblings are constructed starting from short seeds, the
seeds are compact representations of the garblings.
%
Although it would be insecure for the evaluator \E\ to receive
\emph{all} seeds from \G, \HK\ show that it \emph{is secure} for her
to receive seeds corresponding to the inactive branches.
Let $\aid$ be the id of the active branch.
\E can reconstruct from seeds the garbling of each inactive branch, use XOR to unstack
the material $\gcir_\aid$, and evaluate $\cir_\aid$ normally.
%
Of course, what is described so far is not secure: the above procedure
implies that \E\ knows \aid, which of course she does not (in general)
know and which she should not learn.


Thus, \HK\ supplies to \E a `bad' seed for the active branch (i.e.,
she receives a seed that is different yet indistinguishable from the
seed used by \G).
From here, the fundamental idea of SGC is that \E
simply \emph{guesses which branch is taken} (she in fact tries  all $b$
branches) and evaluates this guessed branch with the appropriately reconstructed material.
For security, the guess is unverifiable by \E. 
Still, when she guesses right, she indeed evaluates the taken branch and
computes valid GC output labels for that branch.
However, when she guesses wrong, she evaluates the branch
with so-called garbage material (material that is a random-looking string, not
an encryption of circuit truth tables), and computes
\emph{garbage output labels} (i.e., labels that are not the encryption
of $0$ or $1$, but are instead random-looking strings).
%
To proceed past the exit of the conditional and continue garbled evaluation, it is necessary to
`collect'  these garbage labels, i.e., to obliviously  discard them in favor of the valid
labels\footnote{Of course, the final output labels of the conditional
  are fresh,  such that they cannot be cross-referenced with those
obtained in branch evaluation.}.


\HK show how to collect garbage without
interaction using a garbled gadget called a \emph{multiplexer}.
%
The multiplexer can by non-interactively and straight-forwardly constructed by \G,
but only if he \emph{knows all possible garbage labels}.
Once this is satisfied, it is easy to see how \G can produce a gadget
(e.g., an appropriately wired set of garbled translation tables)
that eliminates garbage and propagates true values.


{\bf \G's Uncertainty.}
It is possible for \G\ to acquire all garbage labels.
\HK achieve this by having \G\ emulate the actions of \E\
 on all inactive branches.
To see how this can be done,
 consider \G's knowledge and uncertainty about the garbled evaluation.
 There are three sources of \G's uncertainty:
\begin{itemize}
  \item Input values to each inactive branch.
    This is the largest source of uncertainty (the number of
    possibilities are exponential in the number of input wires), but
    the easiest to handle.  \HK introduce a simple trick:
    they add an additional garbled gadget, the \emph{demultiplexer},
    that `zeros out' the wires into the inactive branches.
    This fully resolves this source of uncertainty.
  \item True index of the actually taken branch, which we denote by \truth.
  \item \E's guess of value of \truth, which we denote by \guess.
\end{itemize}

In total, there are $b^2$ possible sets of labels ($b(b-1)$ garbage sets of labels and $b$ valid sets of labels)  that the evaluator
could arrive at: one for each combination of $(\truth,\guess)$, the  actually taken  and
the incorrectly guessed branches.
%


To acquire all possible garbage labels (such that he can build the multiplexer
for garbage collection), the \HK generator assumes all-zero inputs for
inactive branches, and emulates ``in his head'' \E's evaluation of all possible (\truth,\guess) combinations.
% must construct all $b(b-1)$ sets of garbage labels,
This requires that \G\ evaluate (i.e. call \Ev) $b(b-1)$ times on garbage material.
This is the source of the $O(b^2)$ computation.




\subsection{Top-level Intuition of the $O(b \log b)$ Stacked Garbling}
\label{sec:intuition}

Our main contribution is reducing the computation associated with branch processing in SGC from $O(b^2)$ to $O(b \log b)$.
To achieve this, we redesign stacking/unstacking and reduce \G's uncertainty regarding \E's garbled evaluation.  
%At a high level, our approach builds on the following key idea:
In this section we provide highest-level intuition, and provide a more detailed review in~\Cref{sec:techOverviewSG}.
%\medskip
 
   Recall from discussion in~\Cref{sec:bsquaredcost} the  sources of \G's uncertainty, which results in $b^2$ garblings inside \G's emulation of \E: $b$ possible values for each variable \truth and \guess ($\truth \in\{0,b-1\}, \guess\in\{0,b-1\})$.
   For each fixed pair  (\truth,\guess), \G has a {\em fully deterministic view} of how \E evaluates the material and exactly which of the garbage it obtains.  \G can then use the corresponding garbage labels in constructing the garbage collector gadget.
   % (\truth,\guess)  combinations inside conditional.
   
   Our main idea  is to ``consolidate''  processing of many (\truth,\guess) pairs by ensuring that \E's execution (garbling calls, reconstructed GCs and output garbage labels) is the {\em same} across these (\truth,\guess) pairs.  Clearly, this would result in corresponding computational savings. 
   

  
  Here is how we approach this.  Wlog, consider a balanced binary tree with $b$ branches as leaves. 
  For each leaf $\ell$, define $i$-th {\em sibling subtree at level $i$} to be the subtree rooted in a sibling of $i$-th node on the path to $\ell$ from tree root.  Thus, each branch has $\lceil \log b \rceil$ sibling subtrees (exactly $\log b$ sibling subtrees in a balanced binary tree).
  
    
 We reduce the number of possible \truth choices  {\em with respect to  \guess}.  For this, we  change the semantics of variable \truth: it now will not mean which of $b$ branches is active; instead \truth will denote in which sibling subtree of \guess the executed branch resides ($\truth=0$ denotes a correct guess).  There are $\log b + 1$ choices for this \truth.  If \G and \E can efficiently process each of these $b\log b$  (\truth,\guess) combinations  (we show that they can!), we achieve the improved $O(b\log b)$ computation cost.
  
  
  
 




\subsection{Our Contributions}
\label{sec:ourContrib}

% Until recently, it was widely believed that a GC
% proportional to the entire program, including parts of the program
% which are entirely discarded due to conditional branching, must
% be transmitted over a network.
% The breakthrough result of~
\HK shows that GC players need not send a GC
proportional to the entire circuit:
instead communication proportional to only the longest program execution
path suffices.
However, this improved communication comes at a cost:
for a conditional $b$ branches, the players
use $O(b^2)$ computation. 

This is a usually a worthwhile trade-off: GC generation speed is usually much higher
than network speeds (cf. our discussion in~\Cref{sec:whentouse}).
However, as the branching factor grows large, computation
can quickly become the bottleneck due to quadratic scaling.
%
Thus, as we argue in~\Cref{sec:motivationHighB},
a more computationally efficient technique
opens exciting possibilities for rich classes of problems.

\medskip
This work presents \ourschemelong, an improvement to SGC that features
improved computation without compromising communication.
Our contributions include:
\begin{itemize}
  \item Improved time complexity.
    % A {\em concretely efficient}  asymptotic
    % improvement to the computational cost of stacked garbling.
    For $b$ branches, \ourschemelong (or
    \ourscheme)  reduces the cost of garbling  from $O(b^2)$ to
    $O(b \log b)$.
  \item Improved space complexity.
    For $b$ branches, our algorithms require $O(\log b)$ space, an
    improvement from \HK's $O(b)$ requirement.
    % , both in terms of memory required for
    % garbling, and in the number of accesses.   This is much more
    % efficient than~\HK...
    \vlad{compare to HK20.  Probably worth
    writing a subsection.}
  \item High \emph{concrete performance}.
   \vlad{concrete numbers, report implementation stats}
  \item
    A formalization in the \cite{CCS:BelHoaRog12} GC framework proved \vlad{I worried about saying under BHR since we must consider topology separately}
    secure under standard assumptions.
    \HK proved SGC secure by assuming the existence of a random oracle.
    We prove security using a standard PRprove security using a
    standard PRF.
    % Like \HK, we leave the processing of low level gates to an
    % underlying garbling scheme \underscheme.
    % Of course, if \underscheme relies on non-standard assumptions, we
    % can still use \underscheme, but acquire
    % assumptions, in which case \ourschemelong will similarly require
    % them.
\end{itemize}



\subsection{When to use \ourschemelong: a high-level costs consideration}
\label{sec:whentouse}
\dave{title? this doesn't seem accurate}
\vlad{proposed candidate}

We now informally discuss a broad question of
practical importance:

\begin{displayquote}
  ``If my program has complex control flow, how can I most efficiently implement it for 2PC?''
\end{displayquote}

To make the question more precise, we assume that `most efficiently' means
`optimized for shortest total wall-clock time'.
% Clearly, since SGC 
Since
(1) GC is often the most practical approach to 2PC,
(2) the GC bottleneck is communication,
(3) `complex control flow' implies conditional behavior, and
(4) SGC improves communication for programs with conditional behavior,
SGC plays an important role in answering this question.
%
Of course, the chosen cryptographic technique is not the only variable
in the optimization space.
Program transformations, such as described
in~\Cref{sec:motivationHighB}, also play a crucial role.
%
Further, these variables are not independent:
some program transformations may lead to a blowup in the number of
branches.
While SGC alleviates the communication overhead of this blowup, the
players must still incur $b \log b$ computational overhead.
%
So choosing which program transformations to apply depends also on the
performance characteristics of the cryptographic scheme.

Despite the fact that the optimization space for total wall-clock time
is complex, we firmly believe the following claim:
using \ourschemelong over standard GC will \emph{almost always improve
  performance.}
The rest of this section argues this claim.

\paragraph{Assumption: computation vs communication.} 
To discuss the best strategy for applying \ourscheme, we must establish
approximate relative costs of GC computation and communication.  Based
on our experiments, as well as on~\cite{XiaoPersonalComm}, a commodity
laptop running a single core can generate GC material at about $3\times$ the
network bandwidth of $1$ Gbps channel.
%
However, if the GC includes conditional branching, it is easy to
engage \emph{many} cores in parallel (handling of conditional branches
is \emph{highly} parallelizable, even when using our more
sophisticated SGC algorithms).
%
Engaging  $8$ virtual cores, a
typical configuration on a mid-range laptop, would imply up to
$\approx 24\times$ higher GC generation speed vs transmission.
%
At the same time, while 1Gbps is a typical speed in the LAN setting, WAN
speeds are much lower, e.g. 100Mbps.
%
Of course, higher-end computing devices, such as desktop CPUs and GPUs
have higher numbers of cores and/or per-core processing power,
resulting in yet higher GC computation-to-transmission ratio.
%
The availability of precomputation (our \G is costlier than \E) is an
additional factor.
% We assume 
% material can be transmitted while it is generated, and the wall clock time is the maximum of generation and transmission times.

Based on the above, we argue that typical settings allow that the
material computing device is (only) $24\times$ faster than the material transmission
device.




\paragraph{Evaluation approach.}  It is natural to approach this
evaluation by considering the {\em baseline circuit} against which we
measure SGC and \ourschemelong\ performance.    That is, our baseline is a
circuit \cir\ with conditionals, to which we apply garbling scheme
directly, and to which we do not apply any program transformations.
We may compare 2PC based on \ourschemelong with  Yao GC, both instantiated
with base scheme half-gates~\cite{EC:ZahRosEva15}.



\paragraph{Rule of thumb: Always apply \ourscheme.}  Assuming the
speed ratio of GC generation/transmission, and with a few caveats
described next, using \ourschemelong\ for branching will {\em always}
improve over standard GC. 

This is easy to see.  Indeed, we evaluate a given \cir\ as-is.  \G
runs a more computationally demanding  process, garbling and
efficiently combining exactly $2 b \log b$ branches.  Consider a
conditional with $b$ branches.  Classic GC will transmit $b$ branches.
During this time, \G could perform $24 b$ branch garblings. Since
\ourschemelong's \G garbles $2\log b$ branches, the point where
computation will cross over to become a bottleneck is $2^{12} = 4096$
branches.  We conclude that applying \ourschemelong\ improves wal clock
time for nearly all reasonable baseline circuits and settings.


\paragraph{Limits on circuit transformations imposed by computational costs.}
Above we have easily  established that \ourschemelong\ is almost always
better than standard GC.   It is much harder to provide heuristics or
a rough suggestions for which circuit transformations (cf.
in~\Cref{sec:motivationHighB}) to apply, and how aggressively should
they be applied in conjunction with \ourschemelong\ secure evaluation.  We
emphasize that our computational performance improvement of SGC opens
a {\em much} wider optimization space than what was possible with the
prior scheme~\HK.  We leave detailed investigation of this direction
as exciting future work.





