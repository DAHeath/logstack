\section{Related Work}\label{sec:relwork}


GC is the most popular and often the fastest approach to secure two-party computation.  Until recently, it was believed that it is necessary
to transmit the entire GC during 2PC, even for conditional branches that
are not taken.  Recent breakthrough work~\HK showed that this folklore belief is false, and that  it is enough to only transmit GC material  proportional to the
longest execution path rather than to the entire circuit.

We focus our comparison to prior work on~\HK, and then review other related work, such as universal circuits and  stacked garbling work.


\paragraph{Comparison with~\HK.}
While the protocol of~\HK is concretely efficient, its quadratic computational cost presents a  limitation in settings where branching factor $b$ is beyond $5$ or, perhaps, $10$.  This estimate is based on the fact that on a commodity laptop and 1Gbps network, GC generation (and also evaluation by \E) is only about $24\times$ faster than its network transmission (cf. discussion in~\Cref{sec:whentouse}).  Even in very favorable for~\HK settings, where communication is {\em much} more constrained than computation, applying \HK to branching with factors beyond $\approx 20$ or $30$ is not likely to improve total computation time.  Our work pushes this boundary significantly, as discussed in~\Cref{sec:whentouse}. \vlad{depending on our numbers, we could cite~\HK relatively low reported performance.}

Above we only discussed the garbling costs in terms of the number of garbled circuits.  Crucially, memory management is a significant performance factor in GC in general, and in particular in~\HK garbling.  Indeed, retrieving an already-garbled material from RAM may take similar or longer time than regarbling from scratch while operating in cache.

In addition to significantly improving computation cost (i.e. number of calls to \Gb\ and \Ev), our approach offers {\em dramatically} better memory utilization, as we discuss in~\Cref{sec:ourContrib}.  Quadratic construction of~\HK requires keeping track of a quadratic number of garbled circuits \vlad{to check}, while only XXX GCs are in memory at a time.  We discuss this aspect in technical detail in~\Cref{sec:memoryEfficiency}.

In sum, as discussed in~\Cref{sec:whentouse,sec:memoryEfficiency}, we essentially eliminate the concern of increased computation due to Stacked Garbling for standard settings (1 Gbps network, commodity laptop computing device), and open the door to the possibility of applying a large class of efficiency-improving program transformations on the evaluated program.





 
 
 
\paragraph{Universal circuits.}
We use cryptographic techniques to reduce the cost of conditionals.  Our focus is the reduction of communication to that corresponding to the longest branch (SGC), while keeping the computational cost low.
A related direction attempts to instead reduce such 2PC cost by choosing alternate Boolean circuit representations.
Universal circuits (UCs) can be \emph{programmed} to implement any circuit in the entire universe of circuits of a given size $n$~\cite{STOC:Valiant76}.
Researchers continue to search for smaller UC constructions;
recent work achieves size $\approx 4.5 n \log n$~\cite{EPRINT:LipMohSad16,EC:KisSch16,AC:GunKisSch17,EPRINT:ZYZL19,EPRINT:AGKS19}.
%
UCs implement conditionals by programming the UC to be the taken branch. 
When the GC generator knows the evaluated branch, the cost of UC programming is free as the generator directly programs the UC; otherwise programming is sent via $O(n\log n)$ OTs.
However, even in the former ``free programming'' case, 
implementing conditionals via UCs usually does not pay off.
Even for branches with only $2^{10}$ gates, the UC is $4.5 \cdot 10 = 45$ times larger than a single branch. 
For typical conditionals with only 2-3 branches, it is cheaper to separately encrypt and send each branch. 

Motivated in part by branching,~\cite{AC:KenKolWil17} proposed a generalization of UCs called \emph{set} universal circuits ($\cal S$-UCs).
An $\cal S$-UC implements a fixed set of circuits $\cal S$ rather than an entire size-$n$ universe.
\cite{AC:KenKolWil17} focuses on the special case where $|\cal S|$ is small, capturing the case where $\cal S$ is a set of conditional branches.
Their approach applies heuristics to embed $\cal S$ into one programmable circuit.
\cite{AC:KenKolWil17} reports the performance of this heuristic for specific sets of circuits, achieving up to $6\times$ GC size reduction for $32$ branches.
However, some sets did not improve over their original representations.  Our work achieves up to $32\times$ GC size reduction for $32$ branches and requires no per-gate overhead, which is significant  in~\cite{AC:KenKolWil17} (about $22$ garbled rows per gate).  
The work of~\cite{AC:Kolesnikov18} (\Cref{review:ZK}) supersedes~\cite{AC:KenKolWil17} when the GC generator knows the taken branch. 


\paragraph{Other related work.}

There is a small number of works that address branching in secure computation, but are not GC/constant-round 2PC.  

Kolesnikov~\cite{AC:Kolesnikov18} was the first to separate the GC {\em material} from the topology and proposed improved GC branching for the case when the generator \G knows the executed branch.  A subsequent work~\cite{EC:HeaKol20} considers a complementary setting where the GC evaluator \E knows the executed branch, and uses it to construct efficient ZK proofs for circuits with branching.
Our work follows the line of work initated by~\cite{AC:Kolesnikov18,EC:HeaKol20}; it is general 2PC and is constant-round.





\ignore{\item The conditional branches are \emph{nested}.
	That is, instead of organizing $b$ branches as a vector, we
	organize them into a binary tree.
	Suppose \E guesses that a particular branch $\cir_i$ is taken,
	while in fact $\cir_j$, a member of a different subtree, is taken.
	The key idea of our approach is that we ensure that when \ev
	garbles the entire subtree containg $\cir_j$ but not $\cir_i$, she
	computes the same material, regardless of $j$.\todo{intuition still unclear}
}
