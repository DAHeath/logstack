\section{Notation and Assumptions}\label{sec:notation}

\paragraph{Notation.}

Our notation is mostly consistent with the notation of~\HK.

\begin{itemize}
  \item Our garbling scheme is called \ourschemelong. We sometimes
    refer to it by the abbreviation \ourscheme, especially when
    referring to its algorithms.
	\item `\G' is the circuit generator. We refer to \G as
	he, him, his, etc.
	\item `\E' is the circuit evaluator. We refer to \E as
	she, her, hers, etc.
\item `$\cir$' is a circuit. \inpsize(\cir) and \outsize(\cir)
  respectively compute
  the number of input/output wires to $\cir$.
  \item $x \mid y$ denotes the concatenation of strings $x$ and
    $y$.
	% \item Lowercase variables, e.g. $x, y$, refer to \emph{unencrypted} wire values.
	% That is, $x$ and $y$ are two different Boolean values held by two different wires.
	% \item Uppercase variables, e.g. $X, Y$, refer to wire labels which are the encryption of a wire value.
	% E.g., $X$ is the encryption of the value $x$.
	% \item This work discusses \emph{projective} garbling schemes, where each wire
	% has two labels. When the corresponding truth value for a label is known, we place the value as a superscript.
	% For example, we use $X^0$ to refer to the encryption of the bit $0$ on wire $x$.
	% Symmetrically, $X^1$ refers to the encryption of bit $1$.
	% Unannotated variables, e.g. $X$, refer to unknown wire encryptions: $X$ could be
	% either $X^0$ or $X^1$, but it is unspecified or unknown in the
	% given context.
	% \item The variables $s, S, S^0, S^1$ refer to the \emph{branch condition wire}. 
	% In the context of a conditional, $s$ decides which branch is taken.
	\item Following SGC terminology introduced by~\cite{AC:Kolesnikov18}, $\mat$ refers to GC \emph{material}.
	Informally, material is just a collection of garbled tables, i.e. the garbling data which, in conjunction with circuit topology and input labels, is used to compute output labels.
\item We use $\nmat$ to denote the size of material, i.e. $\nmat =
  |\mat|$.
	% In standard garbling schemes, material is a vector of encrypted gate tables.
	% In this work, material can be encrypted gate tables or the XOR stacking of material from different branches.
	% In our work, material does not include the circuit topology or labels.
	% \item In the context of a conditional, subscripts $0$ and $1$
	% associate values with the first (resp.
	% second) branch. For example, $\mat_0$ is the material
	% for branch $\cir_0$.
	% \item The variables $n$ and $m$ respectively denote the number of input
	% wires and output wires of a given circuit.
	\item Variables that represent vectors are denoted  
	 $\vec{x}$.
	We index vectors using bracket notation: $\vec{x}[0]$ accesses the $0$th index of $\vec{x}$.
\item In this work, we extensively use binary trees.
  Suppose $t$ is such a tree. We use subscript notation $t_i$ to denote the
  $i$th leaf of $t$.
  We use pairs of indexes to denote internal nodes of the tree.
  I.e., $t_{i, j}$ is the root of the subtree containing the leaves
  $t_i .. t_j$. $t_{i,i}$ (i.e. the node
  containing only $i$) and $t_i$ both refer to the leaf: $t_{i,i} =
  t_i$.
  It is sometimes convenient to refer to a (sub)tree by its root node
  $\node_{i, j}$ or, when clear from context, simply by $\node$.
	% \item We work with explicit pseudorandom seeds.
	% % We write $a \drawnfrom{S} A$ to indicate that we pseudorandomly draw a value from the domain $A$ using the seed $S$ as a source of randomness and store the result in $a$.
	% When a seed $S$ is used to draw multiple values, we assume that
	% each draw uses a nonce to ensure independent randomness.
	% For simplicity, we leave the counter implicit.
	\item We write $a \drawnfrom{} S$ to denote that $a$ is drawn
    uniformly from the set $S$.
	\item $\indist$ denotes computational indistinguishability.
	\item $\kappa$ denotes the computational security parameter and can be understood as the length of encryption keys (e.g. 128).
	% \item $\lambda$ denotes the empty string.
\end{itemize}

In this work, we evaluate GCs  with input labels that are generated independently of the GC material and do not match the GC.
We call such labels {\em garbage labels}.
During GC evaluation,  garbage labels propagate to the output wires and must eventually  be obliviously dropped in favor of valid labels.
We call the process of canceling out output garbage labels {\em garbage collection}.

\paragraph{Assumptions.}  
\ourschemelong is secure in the standard model. However, higher
efficiency of both the underlying scheme \underscheme and of our garbled gadgets can be achieved under the RO assumption.  Our implementation uses half-gates as \underscheme, and relies on random oracle (RO).


%We assume a random oracle (RO). 
