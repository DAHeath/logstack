
\section{Technical Overview of Our Approach}



the following is to move to technical intuition maybe.
\begin{itemize}
	
	\item The conditional branches are \emph{nested}.
	That is, instead of organizing $b$ branches as a vector, we
	organize them into a binary tree.
	Suppose \E guesses that a particular branch $\cir_i$ is taken,
	while in fact $\cir_j$, a member of a different subtree, is taken.
	The key idea of our approach is that we ensure that when \ev
	garbles the entire subtree containg $\cir_j$ but not $\cir_i$, she
	computes the same material, regardless of $j$.\todo{intuition still unclear}
	\item The multiplexer is \emph{not stacked}.
	The careful reader familiar with \cite{EPRINT:HeaKol20b} may be
	confused about our key idea: Stacked Garbling first presents a
	recursive binary tree approach to branching that they later
	discard in favor of a more efficient vector approach.
	So why is our binary tree approach better?
	The problem with \cite{EPRINT:HeaKol20b}'s recursive construction
	is that the evaluator recursively computes the multiplexer for
	nested sub-conditionals.
	However, doing so leads to a recursive emulation whereby \E
	emulates herself (and hence \G emulates himself as well).
	This recursion leads to quadratic cost for both players.
	The way out is to treat the multiplexer separately, and to opt not
	to stack it.
	If multiplexers are not stacked, then \E need not compute them.
\end{itemize}


% At a high level, the circuit generator, i.e. the player who constructs
% the material,





