\begin{abstract}
  Secure two party computation (2PC) of arbitrary programs can be
  efficiently achieved using garbled circuits (GC).
  Until recently, it was widely believed that a GC
  proportional to the entire program, including parts of the program
  which are entirely discarded due to conditional branching, must
  be transmitted over a network.
  Recent work shows that this belief is \emph{false}, and that instead
  an encryption proportional only to the longest program execution
  path suffices (Heath and Kolesnikov, CRYPTO 20).
  Although this recent work reduces the communication needed to evaluate
  conditionals, it \emph{increases} the amount of computation
  performed by the players.
  For a circuit with conditional branching factor $b$, the players
  consume $O(b^2)$ computation.

  We extend the work of Heath and Kolesnikov by reducing
  the computation overhead from $O(b^2)$ to $O(b \log b)$.
  We observe that the primary cause of the increased computation is the
  oblivious collection of \emph{garbage labels} that emerge during the
  evaluation of inactive conditional branches.
  Garbage is collected by a \emph{multiplexer} component whose
  instantiation is costly to generate.
  At a high level, our extension carefully places the multiplexer such
  that we avoid quadratic behavior.

  TODO are we doing this?
  With this extension in place, we then study the concrete constants
  associated with stacked garbling and with our extension.
  We demonstrate that the cheapest stacking strategies involve
  compromising between so-called \emph{vectorized} branching and
  nested branching.
\end{abstract}
