\section{Lower Bound}\label{sec:lowerbound}
\begin{theorem}
Consider a circuit with $b$ branches.  Suppose \Eval\ and \Gen\ issue only
$O(b)$ calls to \Gb. If \Gen\ sends to \Eval\ the stacking of material from
the $b$ branches, then \Gen\ must issue $\Omega(b^2)$  calls to \Ev.
\end{theorem}
\begin{proof}
Consider an arbitrary execution where branch $b_t$  is taken. \Eval\
garbles each branch at least once (for security), but beyond this the
scheme may instruct \Eval\ to garble each branch an arbitrary number of
times, so long as there are only $O(b)$ total garblings.
Recall,
\Eval\
must garble the non-taken branches (i.e., each branch $b_{i\neq t}$ using a
`good' seed (in order to correctly unstack material to evaluate $b_t$).
Thus, \Eval\ garbles  $b-1$ `good' seeds, one for each branch
$b_{i\neq t}$,
and garbles  $O(b)$ total seeds.

\Eval\ uses these garblings to
unstack $b$ materials: one per `guessed' branch. That is, for each
branch $b_j$, \Eval\ selects one garbling of every other branch
$b_{k\neq j}$, XORs these garblings with the stacked material, and
evaluates $b_j$ on the unstacked material. Recall that in the case of $b_t$,
%
\Eval\ must unstack using the $b-1$ `good' materials (or else the scheme
is not correct). When a garbling is used to evaluate a branch in this
manner, we henceforth simply say that the material is \emph{used}.
Other than using all `good' materials to evaluate the taken branch, a
priori there are no other restrictions on how these $O(b)$ garblings
are used.
%
Said another way, each branch has a `pool' of garblings that are
given, via seed, to \Eval. The garbling scheme instructs \Eval\ to
XOR together $b-1$ of these garblings (one from each pool) to evaluate each
branch.

\begin{itemize}
  \item \textbf{Claim 1.} At most a sub-linear number of branches have more than
    $O(1)$ garblings.
    \begin{subproof}
      By pigeonhole principle.
      If this were not the case, then there
      would be more than $O(b)$ total garblings.

      Informally, this claim says that the pools of garblings are
      generally small, because if they were `too large', then the
      number of garblings would exceed the allowed number of calls to
      \Gb.
    \end{subproof}

  \item \textbf{Claim 2.}
    All except a sub-linear number of branches have a garbling that is
    used $\Omega(b)$ times.
    \begin{subproof}
      By Claim 1 and pigeonhole principle. To evaluate each guessed
      branch, \Eval\ uses one garbling from every other branch.  Thus,
      \Eval\ uses $b-1$ total garblings of each branch. If a branch
      has only $O(1)$ available garblings, at least one of them must
      be chosen by \Eval\ $\Omega(b)$ times. Claim 1 indicates that all
      except a sub-linear number of branches have this property.

      Said less formally, since \Eval\ draws from each pool a `large number'
      (i.e., linear number) of times (once per evaluated branch),
      if a pool is `small', \Eval\ must choose at least one item  a
      large number of times. Since most pools are small, she draws
      some garbling a large number of times from most pools.
    \end{subproof}

  \item \textbf{Claim 3.} At most a sub-linear number of
`good' garblings are used only a sub-linear number of times.
  \begin{subproof}
    Suppose not.
    That is, suppose that $\Omega(b)$ `good' garblings are used only a
    sub-linear number of times.
    Then there exists a distinguisher with a non-negligible advantage
    in identifying the taken branch.
    Consider those branches with
    only $O(1)$ garblings (by Claim 1, almost all branches have this
    property). By the pigeonhole principle, a constant fraction of
    these branches have a `good' garbling used only a sub-linear
    number of times. Consider one of these branches. The distinguisher
    can uniformly sample from the sub-linearly used garblings of this
    branch, and guess that this garbling is the `good' one. Since one
    of the constant number of garblings must be used $\Omega(b)$ times
    (Claim 2), this gives the distinguisher a constant advantage in
    distinguishing which garbling is the `good' one. Recall that the
    distinguisher wishes to guess which branch is taken, and that the
    taken branch uses the `good' garbling. If the distinguisher rules
    out those branches that do not use the guessed `good' garbling,
    then it has a constant advantage in indeed ruling out a non-taken
    branch.

    Less formally, this proof observes the fact that if a large number
    of the `good' garblings are used a small number of times, an
    adversarial evaluator can simply look at which garblings are used
    only a small number of times to help her determine which seeds are
    good.
    %
    This strategy gives constant advantage because most pools (1) have
    one `good' garbling, (2) have only constant elements, and (3) have
    at least element with `high usage'.
    %
    Determining which garblings are `good' gives information about which
    branch is taken, since only the taken branch uses all the `good'
    garblings.
  \end{subproof}
\end{itemize}

By Claim 3, all but a sub-linear number of `good'
garblings are used $\Omega(b)$ times. That is, the `good' garblings are used a
total of $\Omega(b^2)$ times. Equivalently, consider the evaluation of each
branch (each of which uses $b-1$ garblings to unstack). Since each of these
evaluations uses only one garbling of each branch, $\Omega(b)$ evaluations
must use $\Omega(b)$ `good' garblings.  Call the execution we have thus far
considered $A$.
Now, consider a different execution where branch
$b_{s\neq t}$  is taken, and call this new execution $B$. Because $b_s$ is
not taken in $A$, the `good' garbling must be given to \Eval\ in $A$. Suppose
that this garbling is used $\Omega(b)$ times (By claim 3, almost all such
`good' garblings are). Now, for security, the `good' garbling of $b_s$ is
not available to \Eval\ in $B$, because $b_s$ is taken in $B$. Thus, those
$\Omega(b)$ branches that were evaluated using the `good' garbling of $b_s$
in $A$
must be evaluated using a different garbling in $B$ (the `good' garbling
simply isn’t available). So, when \Eval\ evaluates these $\Omega(b)$ branches
in $A$ and when she evaluates the same branches in $B$, she will compute
different per-branch output labels. By extending this argument to
consider all pairs of possibly executed branches, there are $\Omega(b^2)$
possible output labels \Eval\ might compute. By the “evaluation axiom”,
\Gen\ can only eliminate output labels by precomputing them himself.
Thus, he must issue $\Omega(b^2)$ calls to \Ev.
\end{proof}


\begin{theorem}
Consider a circuit with $b$ branches.  Suppose \Gen\ issues only
$O(b)$ calls to \Ev. If \Gen\ sends to \Eval\ the stacking of material from
the $b$ branches, then \Gen\ must issue $O(b^2)$  calls to \Gb.
\end{theorem}
\begin{proof}

\textbf{Claim 1.} At most a sub-linear number of
`good' garblings are used only a sub-linear number of times.
  \begin{subproof}
    Suppose not.
    That is, suppose that $O(b)$ `good' garblings are used only a
    sub-linear number of times.
    Then there exists a distinguisher with a non-negligible advantage
    in identifying the taken branch.
    Consider those branches with
    only $O(1)$ garblings (by Claim 1, almost all branches have this
    property). By the pigeonhole principle, a constant fraction of
    these branches have a `good' garbling used only a sub-linear
    number of times. Consider one of these branches. The distinguisher
    can uniformly sample from the sub-linearly used garblings of this
    branch, and guess that this garbling is the `good' one. Since one
    of the constant number of garblings must be used $O(b)$ times
    (Claim 2), this gives the distinguisher a constant advantage in
    distinguishing which garbling is the `good' one. Recall that the
    distinguisher wishes to guess which branch is taken, and that the
    taken branch uses the `good' garbling. If the distinguisher rules
    out those branches that do not use the guessed `good' garbling,
    then it has a constant advantage in indeed ruling out a non-taken
    branch.

    Less formally, this proof observes the fact that if a large number
    of the `good' garblings are used a small number of times, an
    adversarial evaluator can simply look at which garblings are used
    only a small number of times to help her determine which seeds are
    good.
    %
    This strategy gives constant advantage because most pools (1) have
    one `good' garbling, (2) have only constant elements, and (3) have
    at least element with `high usage'.
    %
    Determining which garblings are `good' gives information about which
    branch is taken, since only the taken branch uses all the `good'
    garblings.
  \end{subproof}
\end{proof}
